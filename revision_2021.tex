% Options for packages loaded elsewhere
\PassOptionsToPackage{unicode}{hyperref}
\PassOptionsToPackage{hyphens}{url}
%
\documentclass[
]{article}
\usepackage{lmodern}
\usepackage{amssymb,amsmath}
\usepackage{ifxetex,ifluatex}
\ifnum 0\ifxetex 1\fi\ifluatex 1\fi=0 % if pdftex
  \usepackage[T1]{fontenc}
  \usepackage[utf8]{inputenc}
  \usepackage{textcomp} % provide euro and other symbols
\else % if luatex or xetex
  \usepackage{unicode-math}
  \defaultfontfeatures{Scale=MatchLowercase}
  \defaultfontfeatures[\rmfamily]{Ligatures=TeX,Scale=1}
\fi
% Use upquote if available, for straight quotes in verbatim environments
\IfFileExists{upquote.sty}{\usepackage{upquote}}{}
\IfFileExists{microtype.sty}{% use microtype if available
  \usepackage[]{microtype}
  \UseMicrotypeSet[protrusion]{basicmath} % disable protrusion for tt fonts
}{}
\makeatletter
\@ifundefined{KOMAClassName}{% if non-KOMA class
  \IfFileExists{parskip.sty}{%
    \usepackage{parskip}
  }{% else
    \setlength{\parindent}{0pt}
    \setlength{\parskip}{6pt plus 2pt minus 1pt}}
}{% if KOMA class
  \KOMAoptions{parskip=half}}
\makeatother
\usepackage{xcolor}
\IfFileExists{xurl.sty}{\usepackage{xurl}}{} % add URL line breaks if available
\IfFileExists{bookmark.sty}{\usepackage{bookmark}}{\usepackage{hyperref}}
\hypersetup{
  pdftitle={Revisión 2021},
  pdfauthor={Lorena Pérez 4926489-9},
  hidelinks,
  pdfcreator={LaTeX via pandoc}}
\urlstyle{same} % disable monospaced font for URLs
\usepackage[margin=1in]{geometry}
\usepackage{color}
\usepackage{fancyvrb}
\newcommand{\VerbBar}{|}
\newcommand{\VERB}{\Verb[commandchars=\\\{\}]}
\DefineVerbatimEnvironment{Highlighting}{Verbatim}{commandchars=\\\{\}}
% Add ',fontsize=\small' for more characters per line
\usepackage{framed}
\definecolor{shadecolor}{RGB}{248,248,248}
\newenvironment{Shaded}{\begin{snugshade}}{\end{snugshade}}
\newcommand{\AlertTok}[1]{\textcolor[rgb]{0.94,0.16,0.16}{#1}}
\newcommand{\AnnotationTok}[1]{\textcolor[rgb]{0.56,0.35,0.01}{\textbf{\textit{#1}}}}
\newcommand{\AttributeTok}[1]{\textcolor[rgb]{0.77,0.63,0.00}{#1}}
\newcommand{\BaseNTok}[1]{\textcolor[rgb]{0.00,0.00,0.81}{#1}}
\newcommand{\BuiltInTok}[1]{#1}
\newcommand{\CharTok}[1]{\textcolor[rgb]{0.31,0.60,0.02}{#1}}
\newcommand{\CommentTok}[1]{\textcolor[rgb]{0.56,0.35,0.01}{\textit{#1}}}
\newcommand{\CommentVarTok}[1]{\textcolor[rgb]{0.56,0.35,0.01}{\textbf{\textit{#1}}}}
\newcommand{\ConstantTok}[1]{\textcolor[rgb]{0.00,0.00,0.00}{#1}}
\newcommand{\ControlFlowTok}[1]{\textcolor[rgb]{0.13,0.29,0.53}{\textbf{#1}}}
\newcommand{\DataTypeTok}[1]{\textcolor[rgb]{0.13,0.29,0.53}{#1}}
\newcommand{\DecValTok}[1]{\textcolor[rgb]{0.00,0.00,0.81}{#1}}
\newcommand{\DocumentationTok}[1]{\textcolor[rgb]{0.56,0.35,0.01}{\textbf{\textit{#1}}}}
\newcommand{\ErrorTok}[1]{\textcolor[rgb]{0.64,0.00,0.00}{\textbf{#1}}}
\newcommand{\ExtensionTok}[1]{#1}
\newcommand{\FloatTok}[1]{\textcolor[rgb]{0.00,0.00,0.81}{#1}}
\newcommand{\FunctionTok}[1]{\textcolor[rgb]{0.00,0.00,0.00}{#1}}
\newcommand{\ImportTok}[1]{#1}
\newcommand{\InformationTok}[1]{\textcolor[rgb]{0.56,0.35,0.01}{\textbf{\textit{#1}}}}
\newcommand{\KeywordTok}[1]{\textcolor[rgb]{0.13,0.29,0.53}{\textbf{#1}}}
\newcommand{\NormalTok}[1]{#1}
\newcommand{\OperatorTok}[1]{\textcolor[rgb]{0.81,0.36,0.00}{\textbf{#1}}}
\newcommand{\OtherTok}[1]{\textcolor[rgb]{0.56,0.35,0.01}{#1}}
\newcommand{\PreprocessorTok}[1]{\textcolor[rgb]{0.56,0.35,0.01}{\textit{#1}}}
\newcommand{\RegionMarkerTok}[1]{#1}
\newcommand{\SpecialCharTok}[1]{\textcolor[rgb]{0.00,0.00,0.00}{#1}}
\newcommand{\SpecialStringTok}[1]{\textcolor[rgb]{0.31,0.60,0.02}{#1}}
\newcommand{\StringTok}[1]{\textcolor[rgb]{0.31,0.60,0.02}{#1}}
\newcommand{\VariableTok}[1]{\textcolor[rgb]{0.00,0.00,0.00}{#1}}
\newcommand{\VerbatimStringTok}[1]{\textcolor[rgb]{0.31,0.60,0.02}{#1}}
\newcommand{\WarningTok}[1]{\textcolor[rgb]{0.56,0.35,0.01}{\textbf{\textit{#1}}}}
\usepackage{graphicx,grffile}
\makeatletter
\def\maxwidth{\ifdim\Gin@nat@width>\linewidth\linewidth\else\Gin@nat@width\fi}
\def\maxheight{\ifdim\Gin@nat@height>\textheight\textheight\else\Gin@nat@height\fi}
\makeatother
% Scale images if necessary, so that they will not overflow the page
% margins by default, and it is still possible to overwrite the defaults
% using explicit options in \includegraphics[width, height, ...]{}
\setkeys{Gin}{width=\maxwidth,height=\maxheight,keepaspectratio}
% Set default figure placement to htbp
\makeatletter
\def\fps@figure{htbp}
\makeatother
\setlength{\emergencystretch}{3em} % prevent overfull lines
\providecommand{\tightlist}{%
  \setlength{\itemsep}{0pt}\setlength{\parskip}{0pt}}
\setcounter{secnumdepth}{-\maxdimen} % remove section numbering

\title{Revisión 2021}
\author{Lorena Pérez 4926489-9}
\date{4/6/2021}

\begin{document}
\maketitle

\hypertarget{explicativo-sobre-la-prueba}{%
\subsection{Explicativo sobre la
prueba}\label{explicativo-sobre-la-prueba}}

Por favor completá tu nombre y CI en el YAML del archivo donde dice
\texttt{author:\ "NOMBRE\ Y\ CI:\ "}. El examen es individual y
cualquier apartamiento de esto invalidará la prueba. Puede consultar el
libro del curso durante la revisión \url{http://r4ds.had.co.nz} así como
el libro de ggplot2 pero no consultar otras fuentes de información.

Los archivos y la información necesaria para desarrollar la prueba se
encuentran en Eva en la pestaña Prueba.

La revisión debe quedar en tu repositorio PRIVADO de GitHub en una
carpeta que se llame Prueba con el resto de las actividades y tareas del
curso. Parte de los puntos de la prueba consisten en que la misma sea
reproducible y tu repositorio de GitHub esté bien organizado.\\
Además una vez finalizada la prueba debes mandarme el archivo pdf y Rmd
a \href{mailto:natalia@iesta.edu.uy}{\nolinkurl{natalia@iesta.edu.uy}} y
por favor recordame tu usuario de GitHub para que sea más sencillo
encontrar tu repositorio, asegurate que haya aceptado la invitación a tu
repositorio y de no ser así enviame nuevamente la invitación a
natydasilva.

Recordar que para que tengas la última versión de tu repositorio debes
hacer \texttt{pull} a tu repositorio para no generar inconsistencias y
antes de terminar subir tus cambios con \texttt{commit} y \texttt{push}.

\textbf{La Revisión vale 130 puntos donde 15 de los puntos son de
reproducibilidad de la misma, organización del repositorio en GitHub,
órden y organización en el código y respuestas.}

\hypertarget{ejercicio-1-90-puntos}{%
\section{Ejercicio 1 (90 puntos)}\label{ejercicio-1-90-puntos}}

\hypertarget{explicativo-sobre-los-datos}{%
\subsection{Explicativo sobre los
datos}\label{explicativo-sobre-los-datos}}

Los datos que vamos a utilizar en este ejercicio son una muestra de
datos a nivel nacional sobre abandono escolar en los años 2016 que ya
utilizamos en la Tarea 2.

\begin{table}[hbpt]
    \centering
    \caption{Variables en \label{tab:VAR} \textbf{muestra.csv}}
    \vspace{0.5cm}
    \begin{tabular}{|l|l|}
        \hline
    \textbf{Variable} &  \textbf{Descripción}  \\
        \hline
        documento & Cédula de Identidad del alumno  \\
        \hline
        nro\_doc\_centro\_educ & Liceo que concurre el alumno en 2016  \\
        \hline
        nombre\_departamento & Nombre del Departamento del centro educativo \\
        \hline
        grupo\_desc & Grupo del alumno en 2016 \\
        \hline
        coberturaT & Cobertura en el primer semestre de 2016  \\
        \hline
        Centro\_Grupo & Liceo y grupo del alumno en 2016 \\
        \hline
        cl    & Cluster - contexto sociocultural del liceo en 1016  \\
        \hline
        Grado\_2016\_UE & Grado del alumno en el 2016 según UE  \\
        \hline
        Grado2013 & Grado del alumno en 2013 según CRM  \\
        \hline
        Grado2014 & Grado del alumno en 2014 según CRM  \\
        \hline
        Grado2015 & Grado del alumno en 2015 según CRM \\
        \hline
        Grado 2016 & Grado del alumno en 2016 según CRM \\
        \hline
        Sexo  & Sexo del alumno \\
        \hline
        Fecha.nacimiento & Fecha de nacimiento del alumno \\
        \hline
        Grupo\_UE\_2017 & Grupo del alumno en 2017 \\
        \hline
        inasistencias & cantidad de inasistencias en el primer semestre de 2016\\
        \hline
        asistencias & cantidad de asistencias en el primer semestre de 2016  \\
        \hline
    \end{tabular}
\end{table}

En el Cuadro \ref{tab:VAR} se presentan las variables en el conjunto de
datos \textbf{muestra.csv}.

\newpage

\begin{enumerate}
\def\labelenumi{\arabic{enumi}.}
\tightlist
\item
  Dentro de tu proyecto de RStudio creá un subdirectorio llamado Datos y
  copiá el archivo muestra.csv. Lee los datos usando alguna función de
  la librería \texttt{readr} y \texttt{here}. \textbf{(5 puntos)}
\end{enumerate}

\begin{Shaded}
\begin{Highlighting}[]
\KeywordTok{library}\NormalTok{(readr)}
\KeywordTok{library}\NormalTok{(here)}
\NormalTok{datos<-}\KeywordTok{read_csv}\NormalTok{(}\KeywordTok{here}\NormalTok{(}\StringTok{"d:/Documents/Rmarkdown/Tareas_STAT_NT/Datos/Prueba"}\NormalTok{, }\StringTok{"muestra.csv"}\NormalTok{))}
\end{Highlighting}
\end{Shaded}

\begin{enumerate}
\def\labelenumi{\arabic{enumi}.}
\setcounter{enumi}{1}
\tightlist
\item
  Utilizando funciones de \texttt{dplyr} transformá la variable Abandono
  para que sea un factor con dos niveles donde el 0 se recodifique a No
  y el 1 a Si. Mostrame el resultado resumido en una tabla con la
  cantidad de observaciones para cada categoría usando \texttt{xtable},
  recordá incluir en el chunk
  \texttt{results=\textquotesingle{}asis\textquotesingle{}}. \textbf{(10
  puntos)}
\end{enumerate}

\begin{Shaded}
\begin{Highlighting}[]
\KeywordTok{library}\NormalTok{(dplyr)}
\KeywordTok{library}\NormalTok{(tidyverse)}
\KeywordTok{library}\NormalTok{(xtable)}
\NormalTok{datos<-datos }\OperatorTok\StringTok{ }\KeywordTok{mutate}\NormalTok{(}\DataTypeTok{Abandono=}\KeywordTok{recode}\NormalTok{(Abandono,}\StringTok{`}\DataTypeTok{0}\StringTok{`}\NormalTok{=}\StringTok{"No"}\NormalTok{,}\StringTok{`}\DataTypeTok{1}\StringTok{`}\NormalTok{=}\StringTok{"Sí"}\NormalTok{))}
\NormalTok{prop<-datos }\OperatorTok\KeywordTok{group_by}\NormalTok{(Abandono)}\OperatorTok\KeywordTok{summarise}\NormalTok{(}\DataTypeTok{n=}\KeywordTok{n}\NormalTok{())}\OperatorTok\KeywordTok{xtable}\NormalTok{()}
\NormalTok{prop}
\end{Highlighting}
\end{Shaded}

\begin{enumerate}
\def\labelenumi{\arabic{enumi}.}
\setcounter{enumi}{2}
\tightlist
\item
  Usando funciones de \texttt{dplyr} respondé ¿Cuál es el porcentaje de
  abandono en Montevideo? \textbf{(10 puntos)}
\end{enumerate}

\begin{Shaded}
\begin{Highlighting}[]
\NormalTok{prop2<-datos }\OperatorTok\KeywordTok{filter}\NormalTok{(nombre_departamento}\OperatorTok{==}\StringTok{"Montevideo"}\NormalTok{)}\OperatorTok\KeywordTok{group_by}\NormalTok{(Abandono)}\OperatorTok\KeywordTok{summarise}\NormalTok{(}\DataTypeTok{proporcion=}\NormalTok{(}\KeywordTok{n}\NormalTok{()}\OperatorTok{/}\KeywordTok{nrow}\NormalTok{(datos))}\OperatorTok{*}\DecValTok{100}\NormalTok{)}\OperatorTok\KeywordTok{xtable}\NormalTok{()}
\KeywordTok{print}\NormalTok{(}\KeywordTok{paste0}\NormalTok{(}\StringTok{"El porcentaje de abandono en Montevideo es "}\NormalTok{,}\KeywordTok{round}\NormalTok{(prop2[}\DecValTok{2}\NormalTok{,}\DecValTok{2}\NormalTok{],}\DecValTok{2}\NormalTok{),}\StringTok{"%"}\NormalTok{))}
\end{Highlighting}
\end{Shaded}

\begin{enumerate}
\def\labelenumi{\arabic{enumi}.}
\setcounter{enumi}{3}
\tightlist
\item
  Reproducí el siguiente gráfico y en vez de ``Gráfico a replicar''
  (\texttt{caption}) debes agregar un título que describa la figura y
  algún comentario interesante de lo que observás en la
  misma.\textbf{(10 puntos) }
\end{enumerate}

\begin{Shaded}
\begin{Highlighting}[]
\NormalTok{prop_depto<-datos }\OperatorTok
\StringTok{  }\KeywordTok{group_by}\NormalTok{(nombre_departamento,Abandono)}\OperatorTok\KeywordTok{summarise}\NormalTok{(}\DataTypeTok{Conteo=}\KeywordTok{n}\NormalTok{())}\OperatorTok\KeywordTok{mutate}\NormalTok{(Proporción=}\KeywordTok{round}\NormalTok{(Conteo}\OperatorTok{/}\KeywordTok{sum}\NormalTok{(Conteo),}\DecValTok{3}\NormalTok{)}\OperatorTok{*}\DecValTok{100}\NormalTok{)}\OperatorTok\KeywordTok{filter}\NormalTok{(Abandono}\OperatorTok{==}\StringTok{"Sí"}\NormalTok{)}\OperatorTok\KeywordTok{xtable}\NormalTok{()}
\NormalTok{prop_depto}\OperatorTok\KeywordTok{ggplot}\NormalTok{(}\KeywordTok{aes}\NormalTok{(}\DataTypeTok{x =}\NormalTok{ Proporción,}\DataTypeTok{y=}\KeywordTok{reorder}\NormalTok{(nombre_departamento, Proporción)))}\OperatorTok{+}\KeywordTok{geom_point}\NormalTok{()}\OperatorTok{+}\KeywordTok{theme}\NormalTok{(}\DataTypeTok{aspect.ratio =} \DecValTok{1}\NormalTok{)}\OperatorTok{+}\KeywordTok{labs}\NormalTok{(}\DataTypeTok{x =} \StringTok{"Proporción", y = "}\NormalTok{Departamentos}\StringTok{")}
\end{Highlighting}
\end{Shaded}

Se observa que el porcentaje de abandono en Flores es nulo, mientras que
Río Negro y San José son los únicos dos departamentos con un porcentaje
de abandono mayor a 10.

\begin{enumerate}
\def\labelenumi{\arabic{enumi}.}
\setcounter{enumi}{4}
\tightlist
\item
  Reproducí el siguiente gráfico realizado solo con los estudiantes que
  abandonaron y en vez de ``Gráfico a replicar'' (\texttt{caption})
  debes agregar un título que describa la figura y algún comentario
  interesante de lo que observás en la misma. La paleta usada es Dark2.
  \textbf{(10 puntos)}
\end{enumerate}

\begin{Shaded}
\begin{Highlighting}[]
\NormalTok{datos<-}\KeywordTok{rename}\NormalTok{(datos,}\StringTok{"Género"}\NormalTok{=}\StringTok{"Sexo"}\NormalTok{)}
\NormalTok{prop_depto_sex<-datos }\OperatorTok\KeywordTok{filter}\NormalTok{(Abandono}\OperatorTok{==}\StringTok{"Sí"}\NormalTok{) }\OperatorTok
\StringTok{  }\KeywordTok{group_by}\NormalTok{(nombre_departamento,Género)}\OperatorTok\KeywordTok{summarise}\NormalTok{(}\DataTypeTok{Conteo=}\KeywordTok{n}\NormalTok{())}\OperatorTok\KeywordTok{mutate}\NormalTok{(Proporción=}\KeywordTok{round}\NormalTok{(Conteo}\OperatorTok{/}\KeywordTok{sum}\NormalTok{(Conteo),}\DecValTok{3}\NormalTok{))}\OperatorTok\KeywordTok{xtable}\NormalTok{()}

\NormalTok{prop_depto_sex}\OperatorTok\KeywordTok{ggplot}\NormalTok{(}\KeywordTok{aes}\NormalTok{(}\DataTypeTok{x =}\NormalTok{ Proporción,}\DataTypeTok{fill =}\NormalTok{ Género,}\DataTypeTok{y=}\KeywordTok{reorder}\NormalTok{(nombre_departamento, }\OperatorTok{-}\NormalTok{Proporción)))}\OperatorTok{+}\KeywordTok{geom_bar}\NormalTok{(}\DataTypeTok{position=}\StringTok{"stack"}\NormalTok{,}\DataTypeTok{stat=}\StringTok{"identity"}\NormalTok{)}\OperatorTok{+}\KeywordTok{scale_fill_brewer}\NormalTok{(}\DataTypeTok{palette=}\StringTok{"Dark2"}\NormalTok{)}\OperatorTok{+}\StringTok{ }\KeywordTok{labs}\NormalTok{(}\DataTypeTok{x =} \StringTok{"Proporción", y = "}\NormalTok{Departamentos}\StringTok{")+theme(legend.position = "}\NormalTok{bottom}\StringTok{")}
\end{Highlighting}
\end{Shaded}

A excepción de Canelones, San José, Soriano y Rocha, más del 50\% de
abandono se corresponde al género Masculino.

\newpage

\begin{enumerate}
\def\labelenumi{\arabic{enumi}.}
\setcounter{enumi}{5}
\tightlist
\item
  Reproducí el siguiente gráfico y en vez de ``Gráfico a replicar''
  (\texttt{caption}) debes agregar un título que describa la figura y
  algún comentario interesante de lo que observás en la misma. La paleta
  usada es Dark2.\textbf{(15 puntos)}
\end{enumerate}

\begin{Shaded}
\begin{Highlighting}[]
\NormalTok{prop_clus<-datos }\OperatorTok
\StringTok{  }\KeywordTok{group_by}\NormalTok{(nombre_departamento,Abandono)}\OperatorTok\KeywordTok{summarise}\NormalTok{(}\DataTypeTok{Conteo=}\KeywordTok{n}\NormalTok{())}\OperatorTok\KeywordTok{mutate}\NormalTok{(Proporción=}\KeywordTok{round}\NormalTok{(Conteo}\OperatorTok{/}\KeywordTok{sum}\NormalTok{(Conteo),}\DecValTok{3}\NormalTok{)}\OperatorTok{*}\DecValTok{100}\NormalTok{)}\OperatorTok\KeywordTok{filter}\NormalTok{(Abandono}\OperatorTok{==}\StringTok{"Sí"}\NormalTok{)}\OperatorTok\KeywordTok{xtable}\NormalTok{()}

\NormalTok{datos}\OperatorTok\KeywordTok{ggplot}\NormalTok{(}\KeywordTok{aes}\NormalTok{(}\DataTypeTok{x =}\NormalTok{ cl,}\DataTypeTok{y=}\KeywordTok{reorder}\NormalTok{(nombre_departamento, Proporción)))}\OperatorTok{+}\KeywordTok{geom_point}\NormalTok{()}\OperatorTok{+}\KeywordTok{theme}\NormalTok{(}\DataTypeTok{aspect.ratio =} \DecValTok{1}\NormalTok{)}\OperatorTok{+}\KeywordTok{labs}\NormalTok{(}\DataTypeTok{x =} \StringTok{"Proporción", y = "}\NormalTok{Departamentos}\StringTok{")}
\end{Highlighting}
\end{Shaded}

\newpage

\begin{enumerate}
\def\labelenumi{\arabic{enumi}.}
\setcounter{enumi}{6}
\tightlist
\item
  Recodificá la variable \texttt{grupo\_desc} que tiene 17 niveles para
  que de 1ro.G.1 a 1ro.G5 sea A de 1ro.G.6 a 1ro G.11 sea B y los
  restantes C. Mostrá el resultado seleccionando la variable
  recodificada y las primeras 6 filas. \textbf{(5 puntos)}
\end{enumerate}

\begin{Shaded}
\begin{Highlighting}[]
\NormalTok{datos<-datos }\OperatorTok\StringTok{ }\KeywordTok{mutate}\NormalTok{(}\DataTypeTok{grupo_desc=}\KeywordTok{ifelse}\NormalTok{(datos}\OperatorTok{$}\NormalTok{grupo_desc}\OperatorTok\StringTok{ }\KeywordTok{c}\NormalTok{(}\StringTok{"1ro. G. 1"}\NormalTok{,}\StringTok{"1ro. G. 2"}\NormalTok{,}\StringTok{"1ro. G. 3"}\NormalTok{,}\StringTok{"1ro. G. 4"}\NormalTok{,}\StringTok{"1ro. G. 5"}\NormalTok{),}\StringTok{"A"}\NormalTok{,}\KeywordTok{ifelse}\NormalTok{(datos}\OperatorTok{$}\NormalTok{grupo_desc}\OperatorTok\StringTok{ }\KeywordTok{c}\NormalTok{(}\StringTok{"1ro. G. 6"}\NormalTok{,}\StringTok{"1ro. G. 7"}\NormalTok{,}\StringTok{"1ro. G. 8"}\NormalTok{,}\StringTok{"1ro. G. 9"}\NormalTok{,}\StringTok{"1ro. G. 10"}\NormalTok{,}\StringTok{"1ro. G. 11"}\NormalTok{),}\StringTok{"B"}\NormalTok{,}\StringTok{"C"}\NormalTok{)))}
\KeywordTok{head}\NormalTok{(datos}\OperatorTok{$}\NormalTok{grupo_desc)}
\end{Highlighting}
\end{Shaded}

\begin{enumerate}
\def\labelenumi{\arabic{enumi}.}
\setcounter{enumi}{7}
\tightlist
\item
  Separá la variable Fecha.nacimiento en tres nuevas variables año, mes
  y dia, para ello usá la función \texttt{separate} de forma que sean
  numéricas. Mostrá el resultado seleccionando las variables documento,
  año, dia y mes con alguna función de \texttt{dplyr} y las primeras 6
  filas.\textbf{(5 puntos)}
\end{enumerate}

\begin{Shaded}
\begin{Highlighting}[]
\NormalTok{datos }\OperatorTok\StringTok{ }\KeywordTok{mutate}\NormalTok{(datos,añ}\DataTypeTok{o=}\KeywordTok{separate}\NormalTok{())}
\end{Highlighting}
\end{Shaded}

\begin{enumerate}
\def\labelenumi{\arabic{enumi}.}
\setcounter{enumi}{8}
\tightlist
\item
  Convertí la variable Fecha.nacimiento como objeto de tipo Date usando
  \texttt{as.Date} de R base y comprobá que la nueva variable
  Fecha.nacimiento es del tipo correcto. \textbf{(5 puntos)}
\end{enumerate}

\begin{Shaded}
\begin{Highlighting}[]
\NormalTok{datos}\OperatorTok{$}\StringTok{`}\DataTypeTok{Fecha nacimiento}\StringTok{`}\NormalTok{<-}\KeywordTok{as.Date}\NormalTok{(datos}\OperatorTok{$}\StringTok{`}\DataTypeTok{Fecha nacimiento}\StringTok{`}\NormalTok{)}
\KeywordTok{class}\NormalTok{(datos}\OperatorTok{$}\StringTok{`}\DataTypeTok{Fecha nacimiento}\StringTok{`}\NormalTok{)}
\end{Highlighting}
\end{Shaded}

\begin{enumerate}
\def\labelenumi{\arabic{enumi}.}
\setcounter{enumi}{9}
\tightlist
\item
  Usando la variable Fecha.nacimiento transformada, se considera que el
  alumno tiene extra-edad leve cuando nace antes del 30 de abril de
  2003. Es decir, tiene un a\textasciitilde no m'as de la edad normativa
  para dicha generaci'on. En base a esta definición creá una nueva
  variable (nombrala extra) que valga 1 si el alumno tiene extra edad
  leve y 0 si no la tiene. Muestra solo el resultado de las primeras 6
  filas. Pista para que la condición tome en cuenta el formato fecha
  podrías usar
  \texttt{as.Date(\textquotesingle{}2003-04-30\textquotesingle{})}.
  \textbf{(10 puntos)}
\end{enumerate}

\begin{Shaded}
\begin{Highlighting}[]
\NormalTok{datos}\OperatorTok{$}\NormalTok{extra<-}\KeywordTok{ifelse}\NormalTok{(datos}\OperatorTok{$}\StringTok{`}\DataTypeTok{Fecha nacimiento}\StringTok{`}\OperatorTok{>}\KeywordTok{as.Date}\NormalTok{(}\StringTok{'2003-04-30'}\NormalTok{),}\DecValTok{1}\NormalTok{,}\DecValTok{0}\NormalTok{)}
\KeywordTok{head}\NormalTok{(datos}\OperatorTok{$}\NormalTok{extra,}\DecValTok{6}\NormalTok{)}
\end{Highlighting}
\end{Shaded}

\begin{enumerate}
\def\labelenumi{\arabic{enumi}.}
\setcounter{enumi}{10}
\tightlist
\item
  Trabajá con un subconjunto de datos que tenga documento, Grado2013,
  Grado2014,Grado2015, Grado2016 y llamale reducida. Con los datos
  reducidos restructuralos para que queden de la siguiente forma usando
  alguna de las funciones del paquete \texttt{tidyr} que vimos en la
  última clase.\textbf{(5 puntos)}
\end{enumerate}

\begin{Shaded}
\begin{Highlighting}[]
\NormalTok{reducida<-datos}\OperatorTok\KeywordTok{select}\NormalTok{(documento, Grado2013, Grado2014,Grado2015,}
\NormalTok{Grado2016 )}
\CommentTok{#reducida<-pivot_longer(reducida,names_to = "Grado",values_to = "Nivel")}
\end{Highlighting}
\end{Shaded}

\begin{Shaded}
\begin{Highlighting}[]
\NormalTok{A tibble}\OperatorTok{:}\StringTok{ }\DecValTok{16}\NormalTok{,}\DecValTok{092}\NormalTok{ x }\DecValTok{3}
\NormalTok{   documento  Grado   Nivel}
       \OperatorTok{<}\NormalTok{int}\OperatorTok{>}\StringTok{  }\ErrorTok{<}\NormalTok{chr}\OperatorTok{>}\StringTok{   }\ErrorTok{<}\NormalTok{chr}\OperatorTok{>}
\StringTok{ }\DecValTok{1}  \DecValTok{52401872}\NormalTok{ Grado2013 4º   }
 \DecValTok{2}  \DecValTok{52401872}\NormalTok{ Grado2014 5º   }
 \DecValTok{3}  \DecValTok{52401872}\NormalTok{ Grado2015 6º   }
 \DecValTok{4}  \DecValTok{52401872}\NormalTok{ Grado2016 }\DecValTok{1}    
 \DecValTok{5}  \DecValTok{54975382}\NormalTok{ Grado2013 5º   }
 \DecValTok{6}  \DecValTok{54975382}\NormalTok{ Grado2014 6º   }
 \DecValTok{7}  \DecValTok{54975382}\NormalTok{ Grado2015 1u   }
 \DecValTok{8}  \DecValTok{54975382}\NormalTok{ Grado2016 }\DecValTok{1}    
 \DecValTok{9}  \DecValTok{54944549}\NormalTok{ Grado2013 4º   }
\DecValTok{10}  \DecValTok{54944549}\NormalTok{ Grado2014 5º   }
\end{Highlighting}
\end{Shaded}

\hypertarget{ejercicio-2-25-puntos}{%
\section{Ejercicio 2 (25 puntos)}\label{ejercicio-2-25-puntos}}

\begin{enumerate}
\def\labelenumi{\arabic{enumi}.}
\tightlist
\item
  En clase vimos distintas visualizaciones para variables categóricas y
  mencionamos como posibles el gráfico de barras y el gráficos de torta.
\end{enumerate}

¿Cuál es el argumento teórico para decir que es siempre preferible un
gráfico de barras a uno de tortas para ver la distribución de una
variable categórica? \textbf{(5 puntos)} Porque al tener muchos niveles,
la visualización se pierde un poco en el pie chart, en cambio usando un
bar chart, aún teniendo muchos niveles, se podrá comparar visualmente
uno con otro. Esto sucede incluso cuando los valores para cada nivel son
parecidos similares, se pierde la visualización con el pie chart.

\begin{enumerate}
\def\labelenumi{\arabic{enumi}.}
\setcounter{enumi}{1}
\tightlist
\item
  ¿Porqué es necesario utilizar \texttt{aspect.ratio\ =\ 1} en un
  diagrama de dispersión? \textbf{(5 puntos)}
\end{enumerate}

Porque nos garantiza que las unidades en ambos ejes, x y y, sean de
igual longitud. Permite una mejor visualización de los datos, dado que
no ``achata'' o ``estira'' el panel del gráfico.

\begin{enumerate}
\def\labelenumi{\arabic{enumi}.}
\setcounter{enumi}{2}
\tightlist
\item
  Generá una función \texttt{compra} que tenga como argumentos un vector
  numérico \texttt{cprod} cantidad de productos a comprar de cada tipo y
  un vector numérico \texttt{cdisp} con la cantidad disponible de dichos
  productos (ambos vectores del mismo largo) que devuelva 1 si se pude
  hacer la compra y 0 en caso contrario. La compra se puede realizar
  siempre que haya stock suficiente para cada producto, es decir que la
  cantidad disponible sea igual o mayor a la cantidad comprada. A su vez
  si alguno de los argumentos no es un vector numérico la función no
  debe ser evaluada y debe imprimir el mensaje ``Argumento no
  numérico''. \textbf{(15 puntos)}
\end{enumerate}

Comprobá que el resultado de la función sea

\texttt{compra(c(1,4,2),\ 1:3)\ =\ 0}

\texttt{compra(c("A","B"),\ 1:3)=\ Argumento\ no\ numérico}

\end{document}
